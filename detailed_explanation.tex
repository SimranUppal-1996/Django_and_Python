VERSION 1
---------------------------------------------------------------------------------
🔢 LIST OF ALL NON PRINTABLE ASCII CHARACTERS (1–31 & 127)
---------------------------------------------------------------------------------
ASCII Code	Character	Description
1	SOH	Start of Header
2	STX	Start of Text
3	ETX	End of Text
4	EOT	End of Transmission
5	ENQ	Enquiry
6	ACK	Acknowledge
7	BEL	Bell (Beep Sound) 🔔
8	BS	Backspace ⌫
9	TAB	Horizontal Tab ↹
10	LF	Line Feed (New Line \n)
11	VT	Vertical Tab
12	FF	Form Feed (New Page)
13	CR	Carriage Return \r
14	SO	Shift Out
15	SI	Shift In
16	DLE	Data Link Escape
17	DC1	Device Control 1
18	DC2	Device Control 2
19	DC3	Device Control 3
20	DC4	Device Control 4
21	NAK	Negative Acknowledge
22	SYN	Synchronous Idle
23	ETB	End of Transmission Block
24	CAN	Cancel
25	EM	End of Medium
26	SUB	Substitute
27	ESC	Escape \x1B
28	FS	File Separator
29	GS	Group Separator
30	RS	Record Separator
31	US	Unit Separator
127	DEL	Delete

⚙️ Why Are They Non-Printable?
These characters were designed for control rather than display.
They were used in old teletypes, printers, and network protocols to control devices.
For example, BEL (chr(7)) makes a beep sound, and ESC (chr(27)) is used in terminal commands.

---------------------------------------------------------------------------------
📌 SequenceMatcher
---------------------------------------------------------------------------------
The SequenceMatcher class (from the difflib module) is used to compare two sequences (like strings, lists, or tuples) and find similarities, 
differences, and matching blocks.

🔍 How It Works

from difflib import SequenceMatcher

It finds the longest matching subsequence between two sequences.
It calculates a similarity ratio (between 0 and 1).
It can identify differences for text comparison.

🛠 Common Use Cases
1️⃣ Find Similarity Between Strings

from difflib import SequenceMatcher

s1 = "hello world"
s2 = "hello there"

similarity = SequenceMatcher(None, s1, s2).ratio()
print(similarity)

🔹 Output:
0.7692307692307693  # (76.9% similarity)

💡 Formula Used:
2M/T
Where:
M = Number of matching characters
T = Total characters in both strings

2️⃣ Find Matching Blocks

matcher = SequenceMatcher(None, "abcdef", "acdfg")
for match in matcher.get_matching_blocks():
    print(match)
    
🔹 Output:
Match(a=0, b=0, size=1)  # 'a'
Match(a=2, b=1, size=2)  # 'cd'
Match(a=5, b=4, size=0)  # End marker

💡 What This Means:

"a" at index 0 in both sequences.
"cd" appears at index 2 in "abcdef" and index 1 in "acdfg".
Last match (size=0) is an end marker.

📌 Why None as the First Argument in SequenceMatcher(None, s1, s2)?
The first argument in SequenceMatcher is isjunk, which is used to define a function that filters out "junk" characters (irrelevant elements, 
like spaces or punctuation).

3️⃣ Find Differences Between Strings

s1 = "apple"
s2 = "appla"

matcher = SequenceMatcher(None, s1, s2)
for opcode in matcher.get_opcodes():
    print(opcode)
    
🔹 Output:
('equal', 0, 4, 0, 4)   # First 4 characters are the same ("appl")
('replace', 4, 5, 4, 5) # Last letter 'e' -> 'a'

💡 What This Means:
equal → "appl" is identical in both.
replace → "e" is replaced by "a".

🎯 Summary
✅ SequenceMatcher helps compare sequences for similarity, matching parts, and differences.
✅ Used in spell-checking, plagiarism detection, and version control (like Git diffs).
✅ .ratio() → Similarity score.
✅ .get_matching_blocks() → Identifies common parts.
✅ .get_opcodes() → Shows differences.

📌 Difference Between cmd /C and cmd /K
Both cmd /C and cmd /K are used to execute commands in Windows Command Prompt, but they behave differently:

Command	What It Does
cmd /C	Runs a command and closes the CMD window immediately after execution.
cmd /K	Runs a command and keeps the CMD window open after execution.

VERSION 2
----------------------------------------------------------------------------------
LIBRARIES USED
----------------------------------------------------------------------------------
📂 1. File Handling & System Utilities

os → Interact with the file system (create directories, check file existence, path operations).
re →	Regular expressions for text processing (cleaning text, removing special characters, etc.).
glob →	Find files matching a pattern (used to get .pdf files from a folder).
base64 →	Encode/Decode binary data (like images) into Base64 format (used for embedding images in HTML).
BytesIO (from io) →	Handles binary data streams in memory (used for image processing before encoding).

📊 2. Data Processing & Machine Learning

numpy →	Used for numerical computations, array operations, and distance calculations.
pandas →	Used for handling dataframes (storing distances between texts for plagiarism detection).
torch →	PyTorch framework for deep learning (used to load BERT for text embeddings).
transformers →	Hugging Face BERT Tokenizer & Model for text processing.

📄 3. Natural Language Processing (NLP)

nltk →	Natural Language Toolkit (NLP) library.
nltk.tokenize.word_tokenize →	Splits text into words (used for text pre-processing).
nltk.corpus.stopwords →	Removes common words like "the", "is", etc.
nltk.stem.PorterStemmer →	Converts words to their root forms (e.g., "running" → "run").
nltk.stem.WordNetLemmatizer →	Similar to stemming but more accurate (e.g., "better" → "good").

🔹 NLTK Downloads:
nltk.download('punkt')      # Tokenization
nltk.download('stopwords')  # Stopword removal
nltk.download('wordnet')    # Lemmatization
nltk.download('omw-1.4')    # WordNet dependencies

📜 4. PDF Processing

pdfplumber →	Extract text from PDF files (used to read documents for plagiarism detection).

🎨 5. Data Visualization

matplotlib.pyplot →	Used to plot heatmaps of distances between text embeddings.
seaborn →	Creates better-looking heatmaps for plagiarism analysis.

🌐 6. Django Web Framework (Back-End)

django.conf.settings →	Access Django settings (e.g., file upload paths).
django.core.files.storage.FileSystemStorage →	Handles file uploads.
django.http.JsonResponse →	Sends JSON responses to the frontend.
django.shortcuts.render →	Renders HTML templates with dynamic data.

🚀 Summary
File Handling → os, re, glob, base64, BytesIO
Data Science & ML → numpy, pandas, torch, transformers
NLP → nltk, pdfplumber
Visualization → matplotlib, seaborn
Django (Web Framework) → django.conf, django.core.files.storage, django.http, django.shortcuts

-------------------------------------------------------------------------------------
inputs = tokenizer(text, return_tensors='pt', truncation=True, padding=True)
-------------------------------------------------------------------------------------
tokenizer(text, ...) → Converts raw text into tokenized input.
return_tensors='pt' → Returns a PyTorch tensor ('pt' stands for PyTorch).
truncation=True → Truncates text if it exceeds the model's max length.
padding=True → Pads shorter texts to ensure uniform input size.

-------------------------------------------------------------------------------------
 with torch.no_grad():
        outputs = model(**inputs)
    return outputs.last_hidden_state.mean(dim=1).squeeze().numpy()
-------------------------------------------------------------------------------------
with torch.no_grad():Disables gradient calculations, saving memory and speeding up inference.
Useful when you don't need to train the model, only extract embeddings.

outputs = model(**inputs):Feeds tokenized inputs into the model.
Returns a dictionary where outputs.last_hidden_state is the main output.
outputs.last_hidden_state.mean(dim=1)

last_hidden_state: has shape (batch_size, seq_length, hidden_dim).
Taking .mean(dim=1) computes the mean embedding across all tokens, creating a single vector per input.
.squeeze().numpy()

.squeeze(): removes unnecessary dimensions (e.g., if the batch size is 1).

.numpy(): converts the PyTorch tensor to a NumPy array.

-----------------------------------------------------------------------------------
text = re.sub(r'[^\w\s]', '', text)
text = re.sub(r'\d+', '', text)
-----------------------------------------------------------------------------------
re.sub(r'[^\w\s]', '', text):Matches any character that is NOT a word character (\w) or whitespace (\s).
Removes punctuation (e.g., !, ?, ,, .).

re.sub(r'\d+', '', text):Matches one or more (+) digits (\d).
Removes numbers.

sub - substitute

-------------------------------------------------------------------------------------------
buffer = BytesIO()
plt.savefig(buffer, format='png', bbox_inches='tight', pad_inches=0, transparent=True)
plt.close()
buffer.seek(0)
--------------------------------------------------------------------------------------------
buffer = BytesIO():Creates an in-memory buffer using BytesIO. This buffer will temporarily store the image data as a binary stream.
plt.savefig(buffer, ...):Saves the plot to the buffer instead of a file.
format='png': specifies the image format (PNG).
bbox_inches='tight': ensures that the plot is tightly cropped, avoiding extra whitespace around the plot.
pad_inches=0: eliminates any padding.
transparent=True: makes the background transparent.
plt.close():Closes the plot to avoid using up memory when working with multiple plots.
buffer.seek(0):Moves the read pointer back to the start of the buffer so that you can read the image data from the start.
     
